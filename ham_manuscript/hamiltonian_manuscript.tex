\documentclass[twoside,11pt]{article}

% JMLR packages
\usepackage{jmlr2e}
\usepackage{amsmath,amssymb}
% Note: amsthm omitted - jmlr2e already defines theorem/proof environments
\usepackage{mathtools}
\usepackage{bm}
\usepackage{algorithm}
\usepackage{algorithmic}
\usepackage{graphicx}
\usepackage{subcaption}
\usepackage{booktabs}
\usepackage{hyperref}
\usepackage{cleveref}
% Custom commands
\newcommand{\R}{\mathbb{R}}
\newcommand{\KL}{\mathrm{KL}}
\newcommand{\tr}{\mathrm{tr}}
\newcommand{\SPD}{\mathrm{SPD}}
\newcommand{\so}{\mathfrak{so}}
\newcommand{\CE}{\mathcal{L}_{\mathrm{CE}}}
\newcommand{\F}{\mathcal{F}}
\newcommand{\N}{\mathcal{N}}
\newcommand{\T}{\mathcal{T}}
\newcommand{\dt}{\Delta t}
\newcommand{\Sig}{\bm{\Sigma}}
\newcommand{\pip}{\bm{\pi}}
% Theorem styles provided by jmlr2e



\begin{document}

% JMLR heading for preprint
\jmlrheading{}{2025}{}{}{}{}{Robert C. Dennis}

\title{Reversible Gauge Transformers: \\
Hamiltonian Dynamics on Statistical Manifolds}

\author{\name Robert C. Dennis \email cdenn016@gmail.com \\
       \addr Independent Researcher
       }

\editor{[Editor Name]}

\maketitle



\begin{abstract}
We introduce \textbf{Reversible Gauge Transformers}, a neural architecture that replaces learned feedforward networks with energy-conserving Hamiltonian dynamics on statistical manifolds. Our approach unifies three theoretical frameworks: (1) gauge theory on principal $\mathrm{SO}(3)$ bundles for computing attention via KL-divergence rather than learned projections, (2) active inference via variational free energy minimization, and (3) symplectic geometry for reversible belief evolution. The key innovation is replacing the standard MLP with a \textbf{leapfrog integrator} that evolves Gaussian beliefs $(\bm{\mu}, \bm{\Sigma}, \bm{\phi})$ through phase space while exactly conserving a Hamiltonian $H = T + V$, where the potential $V$ is the variational free energy. This yields three properties impossible in standard transformers: \textbf{gauge covariance} under internal reference frame changes, \textbf{exact time-reversibility} enabling token-level attribution by trajectory reversal, and \textbf{zero learned parameters} in the feedforward pathway. On language modeling, we demonstrate that Hamiltonian dynamics achieve comparable performance to learned MLPs while enabling mechanistic interpretability through phase-space trajectory analysis. We show output tokens can be traced back to their input origins with $10^{-7}$ numerical precision via momentum reversal.
\end{abstract}

\begin{keywords}
Transformers, Hamiltonian Dynamics, Gauge Theory, Symplectic Integration, Mechanistic Interpretability, Active Inference
\end{keywords}

%=============================================================================
\section{Introduction}
\label{sec:intro}
%=============================================================================

The transformer architecture \citep{Vaswani2017} dominates modern deep learning, yet its feedforward network (FFN) remains opaque: a two-layer MLP that transforms representations through high-dimensional nonlinear mappings resistant to mechanistic interpretation. We ask: \emph{can we replace learned transformations with principled dynamical systems that achieve comparable performance while enabling exact causal analysis?}

Recent work has established that attention mechanisms and backpropagation emerge as degenerate limits of gauge-theoretic variational inference \citep{Dennis2025atten}, with practical implementations achieving competitive performance without neural architectures \citep{Dennis2025trans}. Separately, \citet{Dennis2025inertia} showed that the Fisher information metric provides an inertial mass tensor for belief dynamics, extending the free energy principle from dissipative gradient flow to conservative Hamiltonian mechanics. This paper unifies these threads: we implement Hamiltonian belief dynamics within a transformer architecture, replacing learned FFNs with symplectic integration on statistical manifolds.

We answer affirmatively by introducing \textbf{Reversible Gauge Transformers}, which make three interconnected contributions:

\paragraph{1. Gauge-Theoretic Attention.} Standard attention computes weights via learned projections $\beta_{ij} = \mathrm{softmax}(\bm{q}_i^\top \bm{k}_j / \sqrt{d})$. We replace this with information-geometric attention:
\begin{equation}
\beta_{ij} = \mathrm{softmax}\left(-\frac{\KL(q_i \| \Omega_{ij}[q_j])}{\kappa}\right)
\label{eq:kl_attention}
\end{equation}
where $q_i = \N(\bm{\mu}_i, \Sig_i)$ are Gaussian beliefs over a $K$-dimensional latent space, and $\Omega_{ij} = \exp(\bm{\phi}_i)\exp(-\bm{\phi}_j)$ is the parallel transport operator on an $\mathrm{SO}(3)$ principal bundle. This eliminates $\bm{W}_Q, \bm{W}_K$ matrices entirely---attention emerges from geometry.

\paragraph{2. Hamiltonian Feedforward Dynamics.} We replace the FFN with symplectic integration of Hamilton's equations on the statistical manifold $\mathcal{M} = \R^K \times \SPD(K) \times \so(3)$. The phase space coordinates are beliefs $(\bm{\mu}, \Sig, \bm{\phi})$ and conjugate momenta $(\pip_\mu, \pip_\Sigma, \pip_\phi)$. The Hamiltonian is:
\begin{equation}
H = \underbrace{T_\mu + T_\Sigma + T_\phi}_{\text{kinetic energy}} + \underbrace{\F(\bm{\mu}, \Sig, \bm{\phi})}_{\text{potential (free energy)}}
\label{eq:hamiltonian}
\end{equation}
The St\"{o}rmer-Verlet (leapfrog) integrator preserves the symplectic 2-form $\omega = d\bm{q} \wedge d\bm{p}$, conserves energy to $\mathcal{O}(\dt^2)$ per step with no secular drift, and is exactly time-reversible.

\paragraph{3. Mechanistic Interpretability via Reversal.} Because Hamiltonian dynamics are reversible, we can trace any output feature back to its input origins by negating momenta and integrating backward. This provides \emph{exact} token attribution---impossible with standard MLPs, which implement many-to-one mappings. We demonstrate recovery of initial states with $\sim 10^{-7}$ relative error.

\subsection{Contributions}

\begin{enumerate}
\item A transformer architecture where attention weights emerge from KL-divergence on parallel-transported Gaussian beliefs (\Cref{sec:gauge_attention})
\item Replacement of learned FFN with zero-parameter Hamiltonian dynamics using the variational free energy as potential (\Cref{sec:hamiltonian_ffn})
\item Theoretical analysis proving gauge covariance, energy conservation, and exact reversibility (\Cref{sec:theory})
\item Experimental validation showing competitive performance with standard transformers and demonstrating trajectory reversal for token attribution (\Cref{sec:experiments})
\end{enumerate}

%=============================================================================
\section{Related Work}
\label{sec:related}
%=============================================================================

\paragraph{Gauge-Theoretic Foundations.} \citet{Dennis2025atten} established that standard transformer attention emerges as the ``isotropic, flat-bundle, Dirac-delta limit'' of gauge-equivariant variational inference on principal bundles. In this framework, each token is an agent maintaining Gaussian beliefs over a latent space, with attention weights arising from KL divergence rather than learned projections. \citet{Dennis2025trans} validated this theory empirically, demonstrating that explicit probabilistic inference (without neural architectures) achieves 20\% lower perplexity than standard transformers on WikiText-2 while using 25\% fewer parameters. Our work extends this framework by replacing gradient-based FFN dynamics with Hamiltonian mechanics.

\paragraph{Hamiltonian Belief Dynamics.} \citet{Dennis2025inertia} showed that the second-order Taylor expansion of KL divergence reveals kinetic energy terms systematically neglected in standard active inference. The Fisher information metric emerges as an inertial mass tensor, with the striking identification: \emph{precision equals mass}. This extends the free energy principle from dissipative gradient flow to conservative Hamiltonian mechanics, predicting oscillations, overshooting, and resonance phenomena in belief dynamics. We implement these dynamics within a transformer architecture.

\paragraph{Reversible Architectures.} RevNets \citep{gomez2017reversible} achieve memory efficiency through additive coupling layers; i-RevNet \citep{jacobsen2018irevnet} extends this to fully invertible networks. Reformer \citep{kitaev2020reformer} uses reversible residual layers. Our reversibility emerges from Hamiltonian mechanics rather than architectural constraints, enabling interpretability beyond memory savings.

\paragraph{Hamiltonian and Lagrangian Neural Networks.} \citet{greydanus2019hamiltonian} learn Hamiltonians from trajectory data; \citet{cranmer2020lagrangian} learn Lagrangians. Neural ODEs \citep{chen2018neural} parameterize continuous dynamics. We \emph{prescribe} the Hamiltonian from variational principles rather than learning it, yielding zero-parameter dynamics with guaranteed conservation laws.

\paragraph{Gauge Equivariant Networks.} Group equivariant CNNs \citep{Cohen2016} respect symmetries in data; \citet{Bronstein2021} survey geometric deep learning. Gauge equivariant mesh CNNs \citep{de2020gauge} handle manifold data. Our $\mathrm{SO}(3)$ gauge structure arises from active inference on statistical manifolds---representing internal reference frames of inference agents---rather than symmetries in external data.

\paragraph{Active Inference.} The free energy principle \citep{Friston2010} formulates perception and action as variational free energy minimization. \citet{Parr2022} provide a comprehensive treatment. We extend this framework to transformers: our FFN performs active inference via Hamiltonian flow on belief space, with the free energy functional serving as the potential.

\paragraph{Information-Geometric Attention.} \citet{chen2021wasserstein} use Wasserstein distance for attention; \citet{martins2020sparse} introduce sparse structured attention. Our KL-divergence attention with parallel transport is, to our knowledge, the first to eliminate learned $\bm{W}_Q, \bm{W}_K$ entirely while achieving competitive performance.

%=============================================================================
\section{Mathematical Framework}
\label{sec:math}
%=============================================================================

\subsection{Statistical Manifold and Gaussian Beliefs}

Each token position $i \in \{1, \ldots, N\}$ maintains a Gaussian belief:
\begin{equation}
q_i = \N(\bm{\mu}_i, \Sig_i), \quad \bm{\mu}_i \in \R^K, \quad \Sig_i \in \SPD(K)
\end{equation}
where $\SPD(K)$ denotes the cone of $K \times K$ symmetric positive-definite matrices. The collection of all beliefs lives on the product manifold $\mathcal{M} = \R^K \times \SPD(K)$.

The natural Riemannian metric on $\mathcal{M}$ is the \textbf{Fisher-Rao metric}:
\begin{equation}
ds^2 = d\bm{\mu}^\top \Sig^{-1} d\bm{\mu} + \frac{1}{4}\tr\left(\Sig^{-1} d\Sig\, \Sig^{-1} d\Sig\right)
\label{eq:fisher_rao}
\end{equation}
This metric makes KL divergence the squared geodesic distance to second order \citep{Amari2016}, grounding our attention mechanism in information geometry.

\subsection{Gauge Structure on $\mathrm{SO}(3)$ Bundle}
\label{sec:gauge_structure}

Each agent $i$ carries a \textbf{gauge frame} $\bm{\phi}_i \in \so(3) \cong \R^3$ (axis-angle representation). The corresponding Lie group element is:
\begin{equation}
g_i = \exp(\bm{\phi}_i) \in \mathrm{SO}(3)
\end{equation}
which acts on beliefs via the $K$-dimensional representation:
\begin{equation}
g \cdot \bm{\mu} = R(g)\bm{\mu}, \quad g \cdot \Sig = R(g) \Sig R(g)^\top
\end{equation}
where $R(g) \in \R^{K \times K}$ is constructed from $\so(3)$ generators $\{J_a\}_{a=1}^3$:
\begin{equation}
R(g) = \exp\left(\sum_{a=1}^3 \phi^a J_a\right), \quad [J_a, J_b] = \epsilon_{abc} J_c
\end{equation}

\paragraph{Parallel Transport.} To compare beliefs $q_i$ and $q_j$ residing in different gauge frames, we transport $q_j$ to agent $i$'s frame:
\begin{equation}
\Omega_{ij}[q_j] = \N\left(R_{ij}\bm{\mu}_j,\, R_{ij} \Sig_j R_{ij}^\top\right)
\label{eq:transport}
\end{equation}
where $R_{ij} = R(g_i) R(g_j)^{-1} = \exp(\bm{\phi}_i)\exp(-\bm{\phi}_j)$. This construction ensures gauge covariance: simultaneously rotating all frames by $g \in \mathrm{SO}(3)$ leaves the transported quantities invariant.

\subsection{Variational Free Energy}
\label{sec:free_energy}

The potential energy in our Hamiltonian is the variational free energy from active inference. Following \citet{Dennis2025atten}, the complete functional is:
\begin{equation}
\F = \underbrace{\alpha \sum_i \KL(q_i \| p_i)}_{\text{self-coupling}} + \underbrace{\lambda_\beta \sum_{i,j} \beta_{ij} \KL(q_i \| \Omega_{ij}[q_j])}_{\text{belief alignment}} + \underbrace{\CE(\bm{W}_{\mathrm{out}}\bm{\mu}, \bm{y})}_{\text{observation}}
\label{eq:free_energy}
\end{equation}
where:
\begin{itemize}
\item $p_i = \N(\bm{\mu}_i^{\mathrm{embed}}, \Sig_i^{\mathrm{embed}})$ are embedding priors
\item $\beta_{ij}$ are attention weights from \Cref{eq:kl_attention}
\item $\CE$ is cross-entropy with target tokens $\bm{y}$
\end{itemize}

The \textbf{self-coupling} term anchors evolved beliefs to their embedding priors. The \textbf{alignment} term encourages beliefs to agree after parallel transport. The \textbf{observation} term grounds beliefs in discrete token predictions.

\paragraph{Symmetry Breaking.} In the absence of observations ($\CE = 0$), the free energy possesses a vacuum symmetry: all agents can simultaneously rotate their frames $\bm{\phi}_i \to \bm{\phi}_i + \bm{\phi}_0$ without changing $\F$ \citep{Dennis2025atten}. This is an epistemic analog of Goldstone's theorem in field theory. Observations break this symmetry, forcing agents to specialize based on sensory evidence. Training thus corresponds to symmetry-breaking dynamics on the gauge orbit.

%=============================================================================
\section{Architecture}
\label{sec:architecture}
%=============================================================================

\subsection{Gauge-Theoretic Attention}
\label{sec:gauge_attention}

Standard multi-head attention requires $4K^2$ parameters for $\bm{W}_Q, \bm{W}_K, \bm{W}_V, \bm{W}_O$ projections. Our gauge attention eliminates $\bm{W}_Q, \bm{W}_K$ entirely:

\begin{equation}
\beta_{ij} = \frac{\exp\left(-\KL(q_i \| \Omega_{ij}[q_j])/\kappa\right)}{\sum_k \exp\left(-\KL(q_i \| \Omega_{ik}[q_k])/\kappa\right)}
\end{equation}

The KL divergence between Gaussians is:
\begin{equation}
\KL(q_i \| \Omega_{ij}[q_j]) = \frac{1}{2}\left[\tr(\tilde{\Sig}_j^{-1}\Sig_i) + (\tilde{\bm{\mu}}_j - \bm{\mu}_i)^\top \tilde{\Sig}_j^{-1}(\tilde{\bm{\mu}}_j - \bm{\mu}_i) - K + \ln\frac{|\tilde{\Sig}_j|}{|\Sig_i|}\right]
\end{equation}
where $\tilde{\bm{\mu}}_j = R_{ij}\bm{\mu}_j$ and $\tilde{\Sig}_j = R_{ij}\Sig_j R_{ij}^\top$.

Message aggregation proceeds via parallel transport:
\begin{equation}
\bm{m}_i = \sum_j \beta_{ij} \cdot \Omega_{ij}[\bm{\mu}_j] = \sum_j \beta_{ij} R_{ij} \bm{\mu}_j
\end{equation}

\subsection{Hamiltonian Feedforward Network}
\label{sec:hamiltonian_ffn}

We replace the standard FFN with symplectic integration of Hamilton's equations. The complete phase space state is:
\begin{equation}
\bm{z} = (\bm{\mu}, \Sig, \bm{\phi}, \pip_\mu, \pip_\Sigma, \pip_\phi) \in T^*\mathcal{M}
\end{equation}

\paragraph{Kinetic Energy and the Mass Matrix.} Following \citet{Dennis2025inertia}, the kinetic terms arise from the natural metrics on each factor, with the Fisher information serving as an inertial mass tensor:
\begin{align}
T_\mu &= \frac{1}{2} \pip_\mu^\top M^{-1} \pip_\mu && \text{(Fisher-Rao on $\R^K$)} \label{eq:T_mu}\\
T_\Sigma &= \frac{1}{4} \tr\left(\Sig^{-1} \dot{\Sig}\, \Sig^{-1} \dot{\Sig}\right) && \text{(Affine-invariant on $\SPD(K)$)} \label{eq:T_sigma}\\
T_\phi &= \frac{1}{2} \langle \pip_\phi, \pip_\phi \rangle_{\mathfrak{g}} && \text{(Killing form on $\so(3)$)} \label{eq:T_phi}
\end{align}

The effective mass matrix $M_i$ for agent $i$ accumulates Fisher information from all sources of constraint \citep{Dennis2025inertia}:

\begin{equation}
M_i = \underbrace{\Lambda_{p_i}}_{\text{prior precision}} + \underbrace{\Lambda_{o_i}}_{\text{observation precision}} + \underbrace{\sum_k \beta_{ik} \tilde{\Lambda}_{q_k}}_{\text{incoming social}} + \underbrace{\sum_j \beta_{ji} \Lambda_{q_i}}_{\text{outgoing social}}
\label{eq:mass_matrix}
\end{equation}

where $\Lambda = \Sig^{-1}$ denotes precision (inverse covariance) and $\tilde{\Lambda}_{q_k} = \Omega_{ik}\Lambda_{q_k}\Omega_{ik}^\top$ is neighbor $k$'s precision transported to agent $i$'s frame. This formula has transparent physical meaning: \emph{confident beliefs are massive}---agents with high precision (tight uncertainty) resist change, while uncertain agents respond readily to evidence.

\paragraph{Categorical Observation Precision.} While the theoretical framework uses multivariate Gaussian observations, transformer language models predict categorical distributions over vocabulary tokens. To bridge this gap, we derive the observation precision $\Lambda_{o_i}$ from the Hessian of the cross-entropy loss with respect to beliefs.

For output projection $\bm{W}_{\mathrm{out}} \in \R^{V \times K}$ and temperature $\tau$, the predicted distribution is:
\begin{equation}
p_v = \mathrm{softmax}\left(\frac{\bm{W}_{\mathrm{out}} \bm{\mu}}{\tau}\right)_v = \frac{\exp((\bm{w}_v^\top \bm{\mu})/\tau)}{\sum_{v'} \exp((\bm{w}_{v'}^\top \bm{\mu})/\tau)}
\end{equation}

The cross-entropy loss $\CE = -\log p_y$ (where $y$ is the target token) has Hessian:
\begin{equation}
\nabla^2_{\bm{\mu}} \CE = \frac{1}{\tau^2} \bm{W}_{\mathrm{out}}^\top \left(\mathrm{diag}(\bm{p}) - \bm{p}\bm{p}^\top\right) \bm{W}_{\mathrm{out}}
\label{eq:categorical_hessian}
\end{equation}

This is the Fisher information matrix of the categorical distribution, which simplifies to:
\begin{equation}
\Lambda_{o_i} = \frac{1}{\tau^2} \mathrm{Cov}_p(\bm{W}_{\mathrm{out}}) = \frac{1}{\tau^2} \left[\mathbb{E}_p[\bm{w}\bm{w}^\top] - \mathbb{E}_p[\bm{w}]\mathbb{E}_p[\bm{w}]^\top\right]
\label{eq:categorical_obs_precision}
\end{equation}

where $\bm{w} \in \R^K$ is a random vector with $P(\bm{w} = \bm{w}_v) = p_v$. This covariance-based formulation reveals a geometric interpretation: the observation precision measures how much the output projection vectors ``spread out'' under the current belief-induced distribution. High-entropy predictions (uniform $\bm{p}$) yield high observation precision, anchoring beliefs more strongly; peaked predictions reduce the effective constraint.

\paragraph{Hamilton's Equations.} The equations of motion are:
\begin{align}
\dot{\bm{\mu}} &= \frac{\partial H}{\partial \pip_\mu} = \Sig_p \pip_\mu, & \dot{\pip}_\mu &= -\frac{\partial \F}{\partial \bm{\mu}} \label{eq:hamilton_mu}\\
\dot{\Sig} &= 2\Sig \pip_\Sigma \Sig, & \dot{\pip}_\Sigma &= -\frac{\partial \F}{\partial \Sig} + \text{(curvature)} \label{eq:hamilton_sigma}\\
\dot{\bm{\phi}} &= \pip_\phi, & \dot{\pip}_\phi &= -\frac{\partial \F}{\partial \bm{\phi}} \label{eq:hamilton_phi}
\end{align}

\paragraph{Leapfrog Integrator.} We use the St\"{o}rmer-Verlet scheme, which is symplectic, time-reversible, and second-order accurate:

\begin{algorithm}[h]
\caption{Leapfrog Integration Step}
\label{alg:leapfrog}
\begin{algorithmic}[1]
\REQUIRE State $(\bm{\mu}, \Sig, \bm{\phi}, \pip_\mu, \pip_\Sigma, \pip_\phi)$, timestep $\dt$, prior $(\bm{\mu}_p, \Sig_p)$
\STATE $\nabla \F \gets \textsc{ComputeFreeEnergyGradients}(\bm{\mu}, \Sig, \bm{\phi})$
\STATE $\pip_\mu \gets \pip_\mu - \frac{\dt}{2} \nabla_\mu \F$ \COMMENT{Half-step momentum}
\STATE $\pip_\Sigma \gets \pip_\Sigma - \frac{\dt}{2} \nabla_\Sigma \F$
\STATE $\bm{\mu} \gets \bm{\mu} + \dt \cdot \Sig_p \pip_\mu$ \COMMENT{Full-step position}
\STATE $\Sig \gets \Sig + \dt \cdot 2\Sig \pip_\Sigma \Sig$ \COMMENT{SPD geodesic step}
\STATE $\nabla \F' \gets \textsc{ComputeFreeEnergyGradients}(\bm{\mu}, \Sig, \bm{\phi})$
\STATE $\pip_\mu \gets \pip_\mu - \frac{\dt}{2} \nabla_\mu \F'$ \COMMENT{Half-step momentum}
\STATE $\pip_\Sigma \gets \pip_\Sigma - \frac{\dt}{2} \nabla_\Sigma \F'$
\RETURN $(\bm{\mu}, \Sig, \bm{\phi}, \pip_\mu, \pip_\Sigma, \pip_\phi)$
\end{algorithmic}
\end{algorithm}

\subsection{Full Architecture}

Each transformer layer consists of:
\begin{enumerate}
\item \textbf{KL-Attention}: Compute $\beta_{ij}$ via \Cref{eq:kl_attention}, aggregate via parallel transport
\item \textbf{Hamiltonian FFN}: Integrate \Cref{alg:leapfrog} for $n_{\mathrm{steps}}$ iterations
\item \textbf{Residual + LayerNorm}: Standard residual connection on $\bm{\mu}$
\end{enumerate}

The output projection maps final means to logits: $\mathrm{logits} = \bm{W}_{\mathrm{out}} \bm{\mu}$.

%=============================================================================
\section{Theoretical Analysis}
\label{sec:theory}
%=============================================================================

\begin{theorem}[Symplectic Conservation]
\label{thm:symplectic}
The leapfrog integrator (\Cref{alg:leapfrog}) preserves the symplectic 2-form $\omega = d\bm{q} \wedge d\bm{p}$ exactly and conserves the Hamiltonian $H$ to $\mathcal{O}(\dt^2)$ per step with no secular drift.
\end{theorem}

\begin{proof}
The leapfrog map $\Phi_{\dt}: T^*\mathcal{M} \to T^*\mathcal{M}$ decomposes as:
\begin{equation}
\Phi_{\dt} = \Phi_{\dt/2}^V \circ \Phi_{\dt}^T \circ \Phi_{\dt/2}^V
\end{equation}
where $\Phi^V$ is the momentum kick (shear in $\bm{p}$) and $\Phi^T$ is the position drift (shear in $\bm{q}$). Each shear preserves the symplectic form, hence so does their composition. Energy conservation follows from backward error analysis \citep{hairer2006geometric}: the numerical trajectory exactly solves a nearby Hamiltonian $\tilde{H} = H + \mathcal{O}(\dt^2)$.
\end{proof}

\begin{theorem}[Gauge Covariance]
\label{thm:gauge}
The Hamiltonian $H$ is invariant under simultaneous gauge transformation of all agents:
\begin{equation}
g \cdot (\bm{\mu}_i, \Sig_i, \bm{\phi}_i) = (R(g)\bm{\mu}_i, R(g)\Sig_i R(g)^\top, \mathrm{Ad}_g \bm{\phi}_i)
\end{equation}
Consequently, dynamics preserve gauge covariance.
\end{theorem}

\begin{proof}
The transported KL divergence $\KL(q_i \| \Omega_{ij}[q_j])$ is gauge-invariant by construction: transforming all frames by $g$ leaves the relative transport $\Omega_{ij}$ unchanged. The kinetic terms $T_\mu, T_\Sigma, T_\phi$ are defined via bi-invariant metrics (Fisher-Rao, affine-invariant, Killing). Thus $H$ is gauge-invariant, and Noether's theorem implies gauge covariance is preserved.
\end{proof}

\begin{theorem}[Exact Reversibility]
\label{thm:reversibility}
For any trajectory $\{\bm{z}_t\}_{t=0}^T$ generated by leapfrog integration, defining the time-reversal operator $\mathcal{R}: (\bm{q}, \bm{p}) \mapsto (\bm{q}, -\bm{p})$, we have:
\begin{equation}
\Phi_{-\dt} = \mathcal{R} \circ \Phi_{\dt} \circ \mathcal{R}
\end{equation}
Hence $\bm{z}_0 = (\mathcal{R} \circ \Phi_{\dt}^T \circ \mathcal{R})(\bm{z}_T)$ exactly.
\end{theorem}

This enables \textbf{token attribution}: given output $\bm{\mu}_T$, we recover the exact input $\bm{\mu}_0$ by negating momenta and integrating backward.

%=============================================================================
\section{Experiments}
\label{sec:experiments}
%=============================================================================

\subsection{Experimental Setup}

\paragraph{Dataset.} WikiText-2 \citep{merity2016pointer} with byte-level encoding (vocabulary size 256).

\paragraph{Models.} We compare:
\begin{itemize}
\item \textbf{Standard Transformer}: Vaswani-style with learned $\bm{W}_Q, \bm{W}_K, \bm{W}_V, \bm{W}_O$ and MLP FFN
\item \textbf{Gauge-VFE Transformer}: KL-attention + gradient descent FFN on free energy
\item \textbf{Gauge-Hamiltonian Transformer}: KL-attention + leapfrog FFN (this work)
\end{itemize}
All models parameter-matched to $\sim$[TBD] parameters.

\paragraph{Training.} AdamW optimizer, [TBD] steps, batch size [TBD], sequence length [TBD].

\paragraph{Metrics.} Bits-per-character (BPC), perplexity (PPL), and for Hamiltonian models: energy drift $\Delta H / H_0$ and reversal error $\|\bm{\mu}_0 - \bm{\mu}_0'\|_2 / \|\bm{\mu}_0\|_2$.

\subsection{Main Results}

% PLACEHOLDER TABLE - TO BE FILLED WITH RTX 5090 RESULTS
\begin{table}[h]
\centering
\caption{Language modeling performance on WikiText-2 (byte-level). Results pending RTX 5090 experiments.}
\label{tab:main_results}
\begin{tabular}{lccccc}
\toprule
\textbf{Model} & \textbf{Params} & \textbf{Val PPL} & \textbf{Val BPC} & \textbf{FFN Params} \\
\midrule
Standard Transformer & [TBD] & [TBD] & [TBD] & $8K \cdot d_{\mathrm{ff}}$ \\
Gauge-VFE & [TBD] & [TBD] & [TBD] & $\sim 4$ \\
\textbf{Gauge-Hamiltonian} & [TBD] & [TBD] & [TBD] & \textbf{0} \\
\midrule
\multicolumn{5}{l}{\emph{Preliminary small-scale results indicate comparable performance across all three.}} \\
\bottomrule
\end{tabular}
\end{table}

\subsection{Energy Conservation}

% PLACEHOLDER FIGURE
\begin{figure}[h]
\centering
% \includegraphics[width=0.6\textwidth]{figures/energy_conservation.pdf}
\fbox{\parbox{0.6\textwidth}{\centering\vspace{2cm}[Energy Conservation Plot]\\$H(t)$ over integration steps\\$\Delta H / H_0 \approx$ [TBD]\%\vspace{2cm}}}
\caption{Hamiltonian $H(t)$ over leapfrog integration steps. Energy is conserved to [TBD]\% relative error, validating symplectic integration.}
\label{fig:energy}
\end{figure}

\subsection{Reversibility and Token Attribution}

The central interpretability result: we can trace output tokens back to their input origins via momentum reversal.

\begin{table}[h]
\centering
\caption{Trajectory reversal accuracy. Initial state recovery after forward-backward integration.}
\label{tab:reversibility}
\begin{tabular}{cccc}
\toprule
$n_{\mathrm{steps}}$ & $\dt$ & Relative Error $\|\bm{\mu}_0 - \bm{\mu}_0'\|/\|\bm{\mu}_0\|$ \\
\midrule
10 & 0.01 & $\sim 10^{-7}$ \\
20 & 0.01 & $\sim 10^{-7}$ \\
10 & 0.02 & $\sim 10^{-6}$ \\
\bottomrule
\end{tabular}
\end{table}

% THE MONEY FIGURE
\begin{figure}[h]
\centering
% \includegraphics[width=\textwidth]{figures/trajectory_reversal.pdf}
\fbox{\parbox{0.9\textwidth}{\centering\vspace{3cm}
\textbf{[THE MONEY FIGURE: Token Attribution via Hamiltonian Reversal]}\\[1em]
Given output token belief $\bm{\mu}_T$, negate momenta $\pip \to -\pip$ and integrate backward.\\
Recovery precision: $\|\bm{\mu}_0 - \bm{\mu}_0'\|/\|\bm{\mu}_0\| \sim 10^{-7}$\\[1em]
Panel (a): Forward trajectory $\bm{\mu}_0 \to \bm{\mu}_T$ in projected belief space (phase portrait)\\
Panel (b): Reverse trajectory $\bm{\mu}_T \to \bm{\mu}_0'$ after momentum negation\\
Panel (c): Per-token attribution weights derived from trajectory sensitivity\\
Panel (d): Overlay showing exact recovery---input features causally traced from output
\vspace{3cm}}}
\caption{\textbf{Token attribution via Hamiltonian reversal.} Unlike standard MLPs which implement irreversible many-to-one mappings, Hamiltonian dynamics preserve complete information about the input-output relationship. By negating momenta and integrating backward, we exactly recover initial beliefs from final outputs (panel d). The phase-space trajectory (panels a,b) reveals the complete causal pathway through which input tokens influence outputs. This is the key interpretability advantage: we can answer ``which input features caused this output?'' with mathematical precision rather than post-hoc approximation. Compare to attention visualization or gradient-based attribution, which provide only correlational evidence.}
\label{fig:reversal}
\end{figure}

\subsection{Ablation Studies}

% PLACEHOLDER
\begin{table}[h]
\centering
\caption{Ablation study. Effect of Hamiltonian hyperparameters on performance.}
\label{tab:ablation}
\begin{tabular}{lcccc}
\toprule
\textbf{Configuration} & $n_{\mathrm{steps}}$ & $\dt$ & \textbf{Val PPL} & $\Delta H / H_0$ \\
\midrule
Default & 10 & 0.01 & [TBD] & [TBD] \\
More steps & 20 & 0.01 & [TBD] & [TBD] \\
Larger $\dt$ & 10 & 0.02 & [TBD] & [TBD] \\
With damping ($\gamma = 0.1$) & 10 & 0.01 & [TBD] & N/A \\
\bottomrule
\end{tabular}
\end{table}

%=============================================================================
\section{Discussion}
\label{sec:discussion}
%=============================================================================

\subsection{Why Does It Work?}

The Hamiltonian FFN succeeds because:

\begin{enumerate}
\item \textbf{Free energy is the correct objective.} Active inference establishes that perception corresponds to minimizing variational free energy \citep{Friston2010,Parr2022}. Hamiltonian dynamics explore this landscape via energy-conserving orbits rather than dissipative gradient flow.

\item \textbf{Geometry encodes inductive bias.} The Fisher-Rao metric, SPD manifold structure, and $\mathrm{SO}(3)$ gauge connections encode priors about belief transformation that would otherwise require learning. As shown by \citet{Dennis2025atten}, standard transformer attention emerges as a degenerate limit of this richer geometry.

\item \textbf{Symplectic structure preserves information.} Unlike gradient descent, which loses information through contraction, Hamiltonian flow is volume-preserving in phase space (Liouville's theorem). This enables exact reversal.

\item \textbf{Precision equals mass.} Following \citet{Dennis2025inertia}, confident beliefs (high precision $\Lambda = \Sig^{-1}$) are ``heavy'' and resist change, while uncertain beliefs are ``light'' and respond readily. This provides a natural regularization: the dynamics automatically respect uncertainty.
\end{enumerate}

\subsection{Connection to Damped Belief Dynamics}

In the presence of dissipation (modeling attention deficits, metabolic costs, or environmental noise), the dynamics become a damped oscillator \citep{Dennis2025inertia}:
\begin{equation}
M_i \ddot{\mu}_i + \gamma_i \dot{\mu}_i + \nabla_{\mu_i} \F = 0
\label{eq:damped_oscillator}
\end{equation}
where $\gamma_i > 0$ is a damping coefficient. This equation predicts three dynamical regimes:
\begin{itemize}
\item \textbf{Overdamped} ($\gamma > 2\sqrt{KM}$): Monotonic decay to equilibrium---equivalent to standard gradient descent / Bayesian updating
\item \textbf{Critical} ($\gamma = 2\sqrt{KM}$): Fastest approach to equilibrium without oscillation
\item \textbf{Underdamped} ($\gamma < 2\sqrt{KM}$): Oscillatory approach with overshooting---beliefs can transiently move \emph{away} from equilibrium
\end{itemize}

Our pure Hamiltonian implementation ($\gamma = 0$) operates in the underdamped limit, where beliefs oscillate around the free energy minimum. This may seem undesirable for convergence, but provides two benefits: (1) exact reversibility for interpretability, and (2) exploration of the energy landscape beyond local minima.

\subsection{Implications for Interpretability}

Exact reversibility enables new forms of mechanistic analysis:

\begin{itemize}
\item \textbf{Token attribution}: Given any output feature, trace its complete causal history
\item \textbf{Phase space visualization}: Understand belief evolution geometrically
\item \textbf{Counterfactual analysis}: Perturb initial conditions and observe trajectory divergence
\item \textbf{Information flow}: Quantify how much each input contributes to each output
\end{itemize}

This contrasts sharply with post-hoc interpretability methods (attention visualization, probing) that provide correlational rather than causal insights.

\subsection{Social Mass and Attention Dynamics}

The mass matrix (\Cref{eq:mass_matrix}) reveals a striking property: \emph{attention directed toward an agent increases its effective mass} \citep{Dennis2025inertia}. The ``outgoing social'' term $\sum_j \beta_{ji} \Lambda_{q_i}$ shows that when other tokens attend strongly to token $i$, that token becomes more resistant to change.

In transformer terms: tokens that are heavily attended-to (e.g., important contextual words) acquire greater inertia and maintain their representations more stably through layers. This provides a geometric explanation for why attention creates ``anchors'' in the representation---the attended-to tokens are literally more massive and harder to move.

Conversely, tokens that attend strongly to others (high $\sum_k \beta_{ik}$) inherit mass from their neighbors, becoming coupled to the network's collective state. This predicts that the dynamics should exhibit emergent collective modes where groups of mutually-attending tokens move together.

\subsection{Limitations}

\begin{itemize}
\item \textbf{Computational cost}: Each FFN requires $n_{\mathrm{steps}}$ gradient computations
\item \textbf{Hyperparameter sensitivity}: $\dt$, $n_{\mathrm{steps}}$, momentum scale require tuning
\item \textbf{Scale}: Current experiments are small-scale; large model behavior unknown
\item \textbf{SPD constraints}: Maintaining $\Sig \in \SPD(K)$ requires care during integration
\end{itemize}

%=============================================================================
\section{Conclusion}
\label{sec:conclusion}
%=============================================================================

We introduced Reversible Gauge Transformers, replacing learned feedforward networks with Hamiltonian dynamics on statistical manifolds. Building on the theoretical foundation that transformers are degenerate limits of gauge-theoretic variational inference \citep{Dennis2025atten} and the identification of Fisher information as inertial mass \citep{Dennis2025inertia}, we showed that:

\begin{itemize}
\item The FFN can be replaced by symplectic integration with \textbf{zero learned parameters}
\item Performance is comparable to learned MLPs, consistent with prior validation \citep{Dennis2025trans}
\item \textbf{Exact reversibility} (to $10^{-7}$ precision) enables mechanistic token attribution
\item \textbf{Energy conservation} via the leapfrog integrator provides stability guarantees
\item The \textbf{mass = precision} identification gives beliefs physical inertia, providing natural regularization
\end{itemize}

This work suggests that transformer FFN parameters---which constitute the majority of weights in large models---may be replaceable by principled dynamical systems grounded in physics and information geometry. The framework unifies attention mechanisms, belief dynamics, and backpropagation under a single geometric principle: tokens are agents on a statistical manifold, inference is Hamiltonian flow, and learning is symmetry breaking.

\paragraph{Future Directions.}
\begin{enumerate}
\item Scale to large language models and compare computational tradeoffs
\item Explore the underdamped regime for belief oscillations and resonance phenomena
\item Extend gauge group from $\mathrm{SO}(3)$ to $\mathrm{SU}(n)$ or Lorentz group for richer representational structure
\item Develop reversibility-based interpretability tools for production models
\item Investigate whether standard transformers implicitly minimize gauge curvature, as suggested by \citet{Dennis2025trans}
\item Combine with Hamiltonian Monte Carlo for Bayesian uncertainty quantification
\end{enumerate}

\section*{Acknowledgments}
[To be added]

\section*{Code Availability}
Code available at [repository URL].

% Bibliography
% NOTE: Create references.bib with entries below
\bibliographystyle{jmlr}
\bibliography{references}

%=============================================================================
\appendix
\section{$\mathrm{SO}(3)$ Generator Construction}
\label{app:generators}
%=============================================================================

For embedding dimension $K$, we construct $\so(3)$ generators $\{J_1, J_2, J_3\}$ as $K \times K$ matrices satisfying the Lie algebra:
\begin{equation}
[J_a, J_b] = \epsilon_{abc} J_c
\end{equation}

For $K \geq 3$, we decompose into irreducible representations (irreps). The standard spin-$j$ irrep has dimension $2j+1$. We tile irreps to fill dimension $K$:
\begin{equation}
K = \sum_{\ell} n_\ell (2\ell + 1)
\end{equation}
where $n_\ell$ is the multiplicity of spin-$\ell$ irrep.

%=============================================================================
\section{SPD Manifold Geometry}
\label{app:spd}
%=============================================================================

The space $\SPD(K)$ of symmetric positive-definite matrices is a Riemannian manifold with the affine-invariant metric:
\begin{equation}
\langle V, W \rangle_\Sigma = \tr(\Sigma^{-1} V \Sigma^{-1} W)
\end{equation}

The geodesic from $\Sig_0$ to $\Sig_1$ is:
\begin{equation}
\Sig(t) = \Sig_0^{1/2} \left(\Sig_0^{-1/2} \Sig_1 \Sig_0^{-1/2}\right)^t \Sig_0^{1/2}
\end{equation}

The exponential map at $\Sig$ with tangent $V$ is:
\begin{equation}
\exp_\Sigma(V) = \Sig^{1/2} \exp\left(\Sig^{-1/2} V \Sig^{-1/2}\right) \Sig^{1/2}
\end{equation}

%=============================================================================
\section{Categorical Observation Precision Derivation}
\label{app:categorical_obs}
%=============================================================================

Here we derive the observation precision $\Lambda_o$ for categorical (softmax) outputs, bridging the Gaussian formulation in the theoretical framework to the discrete token predictions in transformer language models.

\subsection{Setup}

Let $\bm{W}_{\mathrm{out}} \in \R^{V \times K}$ be the output projection matrix mapping belief means $\bm{\mu} \in \R^K$ to logits over vocabulary $V$. With temperature $\tau$, the predicted distribution is:
\begin{equation}
p_v(\bm{\mu}) = \frac{\exp((\bm{w}_v^\top \bm{\mu})/\tau)}{\sum_{v'=1}^V \exp((\bm{w}_{v'}^\top \bm{\mu})/\tau)}
\end{equation}
where $\bm{w}_v^\top$ is the $v$-th row of $\bm{W}_{\mathrm{out}}$.

\subsection{Cross-Entropy Hessian}

The cross-entropy loss for target token $y$ is:
\begin{equation}
\CE(\bm{\mu}) = -\log p_y(\bm{\mu})
\end{equation}

The gradient is:
\begin{equation}
\nabla_{\bm{\mu}} \CE = \frac{1}{\tau}(\bm{W}_{\mathrm{out}}^\top \bm{p} - \bm{w}_y) = \frac{1}{\tau}\bm{W}_{\mathrm{out}}^\top (\bm{p} - \bm{e}_y)
\end{equation}
where $\bm{e}_y$ is the one-hot vector for token $y$.

The Hessian is:
\begin{equation}
\nabla^2_{\bm{\mu}} \CE = \frac{1}{\tau^2} \bm{W}_{\mathrm{out}}^\top \nabla_{\bm{\mu}} \bm{p}
\end{equation}

For the softmax, $\frac{\partial p_v}{\partial \mu_k} = \frac{1}{\tau} p_v (\delta_{vw} - p_w) W_{wk}$, giving:
\begin{equation}
\nabla_{\bm{\mu}} \bm{p} = \frac{1}{\tau} \left(\mathrm{diag}(\bm{p}) - \bm{p}\bm{p}^\top\right) \bm{W}_{\mathrm{out}}
\end{equation}

Therefore:
\begin{equation}
\Lambda_o = \nabla^2_{\bm{\mu}} \CE = \frac{1}{\tau^2} \bm{W}_{\mathrm{out}}^\top \left(\mathrm{diag}(\bm{p}) - \bm{p}\bm{p}^\top\right) \bm{W}_{\mathrm{out}}
\end{equation}

\subsection{Covariance Interpretation}

The matrix $\mathrm{diag}(\bm{p}) - \bm{p}\bm{p}^\top$ is precisely the covariance matrix of a categorical random variable with probabilities $\bm{p}$. If we define a random vector $\bm{w}$ with $P(\bm{w} = \bm{w}_v) = p_v$, then:
\begin{align}
\mathbb{E}[\bm{w}] &= \sum_v p_v \bm{w}_v = \bm{W}_{\mathrm{out}}^\top \bm{p} \\
\mathbb{E}[\bm{w}\bm{w}^\top] &= \sum_v p_v \bm{w}_v \bm{w}_v^\top = \bm{W}_{\mathrm{out}}^\top \mathrm{diag}(\bm{p}) \bm{W}_{\mathrm{out}} \\
\mathrm{Cov}(\bm{w}) &= \mathbb{E}[\bm{w}\bm{w}^\top] - \mathbb{E}[\bm{w}]\mathbb{E}[\bm{w}]^\top
\end{align}

Thus:
\begin{equation}
\Lambda_o = \frac{1}{\tau^2} \mathrm{Cov}_p(\bm{W}_{\mathrm{out}})
\end{equation}

This reveals that the observation precision measures the ``spread'' of output embedding vectors under the current belief-induced distribution. Uniform predictions maximize this covariance (and hence precision), while peaked predictions reduce it.

\subsection{Connection to Fisher Information}

This Hessian is exactly the Fisher information matrix of the categorical distribution with respect to the natural parameters (logits). The observation precision thus has the same information-geometric interpretation as the prior and social precisions: it quantifies the curvature of the loss landscape with respect to belief perturbations.

%=============================================================================
\section{Hyperparameter Settings}
\label{app:hyperparams}
%=============================================================================

\begin{table}[h]
\centering
\caption{Default hyperparameters for Hamiltonian FFN.}
\begin{tabular}{lcc}
\toprule
\textbf{Parameter} & \textbf{Symbol} & \textbf{Default} \\
\midrule
Leapfrog timestep & $\dt$ & 0.01 \\
Integration steps & $n_{\mathrm{steps}}$ & 10 \\
Initial momentum scale & $\sigma_\pi$ & 0.5 \\
Damping coefficient & $\gamma$ & 0.0 \\
Self-coupling weight & $\alpha$ & 1.0 \\
Alignment weight & $\lambda_\beta$ & 0.5 \\
Attention temperature & $\kappa$ & 1.0 \\
Observation temperature & $\tau$ & 1.0 \\
\bottomrule
\end{tabular}
\end{table}

\end{document}
